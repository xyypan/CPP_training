\documentclass{beamer}
\usepackage[utf8]{inputenc}
\usepackage{listings}
\usepackage{verbatim}

\usepackage{graphicx}
\graphicspath{ {./} }
\usepackage{xspace}
\usepackage{textcomp}

\definecolor{light-gray}{gray}{0.85}

\usetheme{Madrid}
\usecolortheme{default}
\setbeamercolor{emph}{fg=blue}
\renewcommand<>{\emph}[1]{%
  {\usebeamercolor[fg]{emph}\only#2{\itshape}#1}%
}

\definecolor{noteBackgroundColor}{RGB}{251,251,119}
\definecolor{noteBorderColor}{RGB}{168,0,54}


\title{Lambdas in C++}
\date{\today}
\author{Pan Xiaoyue}
\institute{Smarkets}


\begin{document}

\frame{\titlepage}

\defverbatim[colored]\lstSimpleLambda{
\begin{lstlisting}[language=C++, basicstyle=\small\ttfamily, keywordstyle=\color{red},tabsize=2, backgroundcolor=\color{light-gray}]
auto simple_lambda = [](int x) { return x+1; };
\end{lstlisting}
}

\defverbatim[colored]\lstInvokeLambda{
\begin{lstlisting}[language=C++, basicstyle=\small\ttfamily, keywordstyle=\color{red},tabsize=2, backgroundcolor=\color{light-gray}]
int x = simple_lambda(100);
std::cout << "x:" << x << "\n";
\end{lstlisting}
}

\defverbatim[colored]\lstI{
\begin{lstlisting}[language=C++, basicstyle=\small\ttfamily, keywordstyle=\color{red},tabsize=2, backgroundcolor=\color{light-gray},commentstyle=\color{blue}\ttfamily]
std::vector<int> a{0,1,2,3,4,5,6};
//lambda1 captures a by value
auto lambda1 = [a](int x) { return a[x]; };
//lambda2 captures a by reference
auto lambda2 = [&a](int x) { a[x]++; };
\end{lstlisting}
}

\defverbatim[colored]\lstImplicitLambda{
\begin{lstlisting}[language=C++, basicstyle=\small\ttfamily, keywordstyle=\color{red},tabsize=2, backgroundcolor=\color{light-gray},commentstyle=\color{blue}\ttfamily]
//Assume we have std::vector<int> a{0,1,2,3,4,5,6};
auto lambda = [&a](int x) { a[x]++; };
\end{lstlisting}
}

\defverbatim[colored]\lstII{
\begin{lstlisting}[language=C++, basicstyle=\small\ttfamily, keywordstyle=\color{red},tabsize=2, backgroundcolor=\color{light-gray},commentstyle=\color{blue}\ttfamily]
struct ExplicitLambda {
	// captures go here
	std::vector<int> &a;
	ExplicitLambda (std::vector<int> &vec): a(vec) {}
	// body of the lambda goes into a callable operator
	void operator() (int x) {
		a[x]++;
	}
};
\end{lstlisting}
}

\defverbatim[colored]\lstWhyNeedCaptures{
\begin{lstlisting}[language=C++, basicstyle=\small\ttfamily, keywordstyle=\color{red},tabsize=2, backgroundcolor=\color{light-gray},commentstyle=\color{blue}\ttfamily]
std::vector<int> a{0,1,2,3,4,5,6};
auto lambda2 = [&a](int x) { a[x]++; };

std::vector<int> indices{0,3};
//std::for_each (lambda2's caller) can't access a
std::for_each(indices.begin(), indices.end(), lambda2);
\end{lstlisting}
}
\defverbatim[colored]\lstCallableObject{
\begin{lstlisting}[language=C++, basicstyle=\small\ttfamily, keywordstyle=\color{red},tabsize=2, backgroundcolor=\color{light-gray}]
a_callable_object();
another_callable_object(args);
\end{lstlisting}
}

\defverbatim[colored]\lstSquirrelStruct{
\begin{lstlisting}[language=C++, basicstyle=\small\ttfamily, keywordstyle=\color{red},tabsize=2, backgroundcolor=\color{light-gray}, showstringspaces=false]
struct Squirrel {
	std::string name;
	int age;
	Squirrel(const std::string &name, int age): 
		name(name), age(age){}
};
Squirrel new_squirrel("Charlie", 5);
\end{lstlisting}
}


\defverbatim[colored]\lstCallSquirrel{
\begin{lstlisting}[language=C++, basicstyle=\small\ttfamily, keywordstyle=\color{red},tabsize=2, backgroundcolor=\color{light-gray}]
Squirrel new_squirrel("Charlie", 5);
new_squirrel();
\end{lstlisting}
}

\begin{frame}
\frametitle{An lambda example}

\begin{itemize}
\item A lambda is an anonymous function, e.g., \lstSimpleLambda

\item Call the lambda: \lstInvokeLambda

\end{itemize}

\end{frame}

\begin{frame}
\frametitle{Exercise: lambda syntax}
1. Write a lambda which prints an integer.
\lstinputlisting [language=C++, basicstyle=\small\ttfamily, keywordstyle=\color{red},tabsize=2, backgroundcolor=\color{light-gray},commentstyle=\color{blue}\ttfamily]
{../code_on_slides/p3.cpp}
2. Generalise it to not just integer
\end{frame}


\begin{frame}
\frametitle{C++'s lambda definition}
Syntax: [ captures ] ( params ) $\rightarrow$ ret \{ body \}
\begin{itemize}
	\item 
	{
	captures: can be both by \emph{value} and by \emph{reference}
		\lstI
	}
	\item {What's the result of \emph{\ttfamily{lambda1(0)}} and \emph{\ttfamily{lambda2(0)}}?}
\end{itemize}

\end{frame}

\begin{frame} 
\frametitle{Why do we need captures?}
\begin{itemize}
	\item 
	{
	Because the code that calls the lambda \emph{may not have access} to the needed variables
	\lstWhyNeedCaptures
	}	
\end{itemize}
\end{frame}

\begin{frame}
\frametitle{Exercise: captures}
\lstinputlisting [language=C++, basicstyle=\small\ttfamily, keywordstyle=\color{red},tabsize=2, backgroundcolor=\color{light-gray},commentstyle=\color{blue}\ttfamily]
{../code_on_slides/p6.cpp}
\end{frame}

\begin{frame}
\frametitle{A little detour: structs and classes}
\begin{itemize}
	\item They are almost the same (diff: default accessibility of member variables and functions)
	\lstSquirrelStruct
	\item Constructors and member initializer lists
\end{itemize}
\end{frame}
%---------------------------------------------------------

\begin{frame}{Callable object}
\begin{itemize}
	\item A callable object is an object that \emph{can be called like a function}, e.g., 
	\lstCallableObject
	\item How do we make an object callable? e.g., 
	\lstCallSquirrel
	\begin{itemize}
		\item By overloading the operator \emph{()}.
	\end{itemize}
\end{itemize}
\end{frame}

\begin{frame} 
\frametitle{Make squirrels callable}
\begin{itemize}
\item Overload \emph{()} in the Squirrel struct
\lstinputlisting [language=C++, basicstyle=\tiny\ttfamily, keywordstyle=\color{red},tabsize=2, backgroundcolor=\color{light-gray},commentstyle=\color{blue}\ttfamily,showstringspaces=false]
{../code_on_slides/p9.cpp}
\end{itemize}
\end{frame}


\begin{frame}
\frametitle{Why do we talk about struct and callable object?}
\begin{itemize}
	\item A lambda is a callable object.
\end{itemize}

The lambda in
\lstImplicitLambda
 is equivalent to
 \centering{\lstII}
\begin{itemize}
	\item {lambda(0) is the same as calling an object of the struct with 0.}
\end{itemize}
\end{frame}

\begin{frame}
\frametitle{Exercise: sort the squirrels}
Write a lambda which can be used by std::sort to sort Squirrel objects by age/name.
\lstinputlisting [language=C++, basicstyle=\tiny\ttfamily, keywordstyle=\color{red},tabsize=2, backgroundcolor=\color{light-gray}, commentstyle=\color{blue}\ttfamily]
{../code_on_slides/p11.cpp}
\end{frame}

\begin{frame}
\frametitle{Exercise: one way to shoot yourself in the foot} 
What does this code print?
\lstinputlisting [language=C++, basicstyle=\small\ttfamily, keywordstyle=\color{red},tabsize=2, backgroundcolor=\color{light-gray},showstringspaces=false]
{../code_on_slides/p12.cpp}
\end{frame}

\begin{frame}
\frametitle{Exercise: if you find the exercises too simple} 
What's the output of this code? 
(Taken from \href{https://en.cppreference.com/w/cpp/language/lambda}{\color{blue}{\underline{https://en.cppreference.com/w/cpp/language/lambda}}}
)
\lstinputlisting [language=C++, basicstyle=\tiny\ttfamily, keywordstyle=\color{red},tabsize=2, backgroundcolor=\color{light-gray}, showstringspaces=false]
{../code_on_slides/p13.cpp}
\end{frame}
\end{document}